\documentclass[11pt,a4paper]{article}
\usepackage[utf8]{inputenc}
\usepackage[T1]{fontenc}
\usepackage[serbian]{babel}
\usepackage[table,xcdraw]{xcolor}
\usepackage{array}
\usepackage{graphicx}
\usepackage{hyperref}
\hypersetup{
	colorlinks,
	citecolor=black,
	filecolor=black,
	linkcolor=black,
	urlcolor=black
}

\renewcommand*\contentsname{Sadržaj}

\begin{document}

\begin{titlepage}

\centering
\textnormal{\large Elektrotehnički fakultet Univerziteta u Beogradu}\\[0.1cm]
\textnormal{\large Prinicipi softverskog inženjerstva}\\[3cm]

\textnormal{\normalsize - Platforma za organizovanje volonterskih akcija -}\\\vspace{-5mm}
\rule{\textwidth}{0.4pt}
{\huge \bfseries Specifikacija scenarija upotrebe:\\ 
REGISTRACIJA KORISNIKA\par}\vspace{-1mm}
\rule{\textwidth}{0.4pt}\\\vspace{1mm}
\textnormal{\large Autor: Nemanja Stojoski}\\[6cm]

\includegraphics[scale=0.5]{logo.jpg}\\
\vfill
\textnormal{\normalsize Verzija 1.0}\\

\end{titlepage}

\tableofcontents

\newpage

\section{Uvod}
\subsection{Rezime}
Definisanje scenarija upotrebe pri registraciji korisnika, sa primerima odgovarajućih HTML stranica.
\subsection{Namena dokumenta i ciljne grupe}
Dokument će koristiti svi članovi projektnog tima u razvoju projekta i testiranju, a može se koristiti i pri pisanju uputstva za upotrebu.
\subsection{Reference}
\begin{enumerate}
    \item Projektni zadatak
    \item Uputstvo za pisanje specifikacije scenarija upotrebe funkcionalnosti
\end{enumerate}
\subsection{Otvorena pitanja}
\begin{center}
\begin{tabular}{| >{\centering\arraybackslash}m{1.9cm} | >{\centering\arraybackslash}m{4.9cm} | >{\centering\arraybackslash}m{4.9cm} |}
\hline
\rowcolor[HTML]{000000} 
{\color[HTML]{FFFFFF} Redni broj } & {\color[HTML]{FFFFFF} Opis } & {\color[HTML]{FFFFFF} Rešenje } \\ \hline
1 & Da li je potrebno vršiti proveru unetih podataka u realnom vremenu (pre klika na dugme za registraciju)? & \\ \hline
 &  &  \\ \hline
 &  &  \\ \hline
 &  &  \\ \hline
\end{tabular}
\end{center}

\newpage

\section{Scenario registracije korisnika}
\subsection{Kratak opis}
Gost sistema može, ukoliko želi, da kreira svoj nalog unošenjem ličnih
podataka. Ti podaci će biti upisani u bazu podataka na osnovu kojih korisnik
kasnije može da pristupa sistemu.
\subsection{Tok događaja}
\subsubsection{\underline{Korisnik se uspešno registruje na sistem}}
\begin{enumerate}
    \item Korisnik odabirom polja \textit{Register} dolazi na stranicu za registraciju
    \item Korisnik unosi obavezne podatke potrebne za registraciju i opciono ostale.
    \item Korisnik pritiska dugme \textit{Register}
    \item Sistem proverava uneto korisničko ime i e-mail adresu sa već postojećim u bazi podataka i vrši evidenciju novog korisnika, ukoliko nema poklapanja
    \item Sistem prebacuje korisnika na početnu stranicu
\end{enumerate}

\subsubsection{U bazi podataka već postoji korisnik sa unetim podacima}
\begin{enumerate}
    \item Akcije 1-3 iz scenarija 2.2.1 su iste
    \item Sistem proverava uneto korisničko ime i e-mail adresu sa već postojećim u bazi podataka i prikazuje poruku o zauzetosti korisničkog imena ili e-mail adrese
\end{enumerate}

\subsubsection{Format e-mail adrese nije odgovarajući}
\begin{enumerate}
    \item Akcija 1 iz scenarija 2.2.1 je ista
    \item Korisnik unosi potrebne podatke i e-mail adresu u neispravnom formatu
    \item Sistem obaveštava korisnika da je e-mail adresa u neispravnom formatu
\end{enumerate}

\subsubsection{Lozinka i potvrda lozinke se ne poklapaju}
\begin{enumerate}
    \item Akcija 1 iz scenarija 2.2.1 je ista
    \item Korisnik unosi potrebne podatke lozinku i različitu potvrdu lozinke
    \item Sistem obaveštava korisnika da se lozinka i potvrda lozinke razlikuju
\end{enumerate}

\subsubsection{Korisnik nije popunio sva polja potrebna za registraciju}
\begin{enumerate}
    \item Akcija 1 iz scenarija 2.2.1 je ista
    \item Korisnik ne popunjava neko od obaveznih polja potrebnih za registraciju
    \item Sistem obaveštava korisnika da nije popunio sva obavezna polja potrebna za registraciju
\end{enumerate}

\subsection{Posebni zahtevi}
Nema
\subsection{Preduslovi}
Korisnik nije ulogovan na sistem.
\subsection{Posledice}
Korisnik je evidentiran u sistemu.

\newpage

\section{Spisak izmena}
\begin{center}
\begin{tabular}{| >{\centering\arraybackslash}m{2cm} | >{\centering\arraybackslash}m{1.3cm} | >{\centering\arraybackslash}m{4.2cm} | >{\centering\arraybackslash}m{4.2cm} |}
\hline
\rowcolor[HTML]{000000} 
{\color[HTML]{FFFFFF} Datum } & {\color[HTML]{FFFFFF} Verzija } & {\color[HTML]{FFFFFF} Opis } & {\color[HTML]{FFFFFF} Autor } \\ \hline
12.03.2016 & 1.0 & Osnovna verzija & Nemanja Stojoski \\ \hline
 &  &  &  \\ \hline
 &  &  &  \\ \hline
 &  &  &  \\ \hline

\end{tabular}
\end{center}

\end{document}