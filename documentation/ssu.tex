\documentclass[11pt,a4paper]{article}
\usepackage[utf8]{inputenc}
\usepackage[T1]{fontenc}
\usepackage[serbian]{babel}
\usepackage[table,xcdraw]{xcolor}
\usepackage{array}
\usepackage{graphicx}
\usepackage{hyperref}
\hypersetup{
	colorlinks,
	citecolor=black,
	filecolor=black,
	linkcolor=black,
	urlcolor=black
}

\renewcommand*\contentsname{Sadržaj}

\begin{document}

\begin{titlepage}

\centering
\textnormal{\large Elektrotehnički fakultet Univerziteta u Beogradu}\\[0.1cm]
\textnormal{\large Prinicipi softverskog inženjerstva}\\[3cm]

\textnormal{\normalsize - Platforma za organizovanje volonterskih akcija -}\\\vspace{-5mm}
\rule{\textwidth}{0.4pt}
{\huge \bfseries Specifikacija scenarija upotrebe:\\ 
REGISTRACIJA KORISNIKA\par}\vspace{-1mm}
\rule{\textwidth}{0.4pt}\\\vspace{1mm}
\textnormal{\large Autor: Stefan Bajić}\\[6cm]

\includegraphics[scale=0.5]{logo.jpg}\\
\vfill
\textnormal{\normalsize Verzija 1.0}\\

\end{titlepage}

\tableofcontents

\newpage

\section{Uvod}
\subsection{Rezime}
Definisanje scenarija upotrebe pri registraciji korisnika, sa primerima odgovarajućih HTML stranica.
\subsection{Namena dokumenta i ciljne grupe}
Dokument će koristiti svi članovi projektnog tima u razvoju projekta i testiranju, a može se koristiti i pri pisanju uputstva za upotrebu.
\subsection{Reference}
\begin{enumerate}
    \item Projektni zadatak
    \item Uputstvo za pisanje specifikacije scenarija upotrebe funkcionalnosti
\end{enumerate}
\subsection{Otvorena pitanja}
\begin{center}
\begin{tabular}{| >{\centering\arraybackslash}m{39mm} | >{\centering\arraybackslash}m{39mm} | >{\centering\arraybackslash}m{39mm} |}
\hline
\rowcolor[HTML]{000000} 
{\color[HTML]{FFFFFF} Redni broj } & {\color[HTML]{FFFFFF} Opis } & {\color[HTML]{FFFFFF} Rešenje } \\ \hline
 &  & \\ \hline
 &  &  \\ \hline
 &  &  \\ \hline
 &  &  \\ \hline
 &  &  \\ \hline
 &  &  \\ \hline
\end{tabular}
\end{center}

\newpage

\section{Scenario registracije korisnika}
\subsection{Kratak opis}
Pretraživanjem Interneta došlo se do zaključka da postojeće volonterske platforme ne pružaju dovoljno fleksibilnosti, kako volonterima, tako i organizatorima dobrovoljačkih akcija. Naša platforma pokušava da na jednostavan način obezbedi glavne funkcionalnosti takvog sistema, sa mogućnošću lakog dodavanja novih funkcionalnosti. Platforma omogućava korisnicima da kreiraju novi događaj, tj. akciju sa opisom, lokacijom, datumom održavanja, potrebnim brojem učesnika, i ostalim relevantnim informacijama. Nakon kreiranja događaja, on postaje vidljiv svim ostalim korisnicima, kako bi se oni mogli prijaviti na isti. Tokom održavanja, i nakon završetka događaja, organizator bi vodio evidenciju o učestvovanju prijavljenih korisnika, na osnovu koje bi se zaslužnim učesnicima dodeljivali virtuelni bedževi.
\subsection{Tok događaja}
Gost sajta može da pregleda predstojeće događaje sortirane po raznim kriterijumima, a ima i mogućnost registracije.
\subsubsection{Korisnik bla bla}
\subsubsection{Korisnik bla bla}
\subsection{Posebni zahtevi}
Nema
\subsection{Preduslovi}
\subsection{Posledice}

\newpage

\section{Spisak izmena}
\begin{center}
\begin{tabular}{| >{\centering\arraybackslash}m{29.25mm} | >{\centering\arraybackslash}m{29.25mm} | >{\centering\arraybackslash}m{29.25mm} | >{\centering\arraybackslash}m{29.25mm} |}
\hline
\rowcolor[HTML]{000000} 
{\color[HTML]{FFFFFF} Datum } & {\color[HTML]{FFFFFF} Verzija } & {\color[HTML]{FFFFFF} Opis izmene } & {\color[HTML]{FFFFFF} Autor } \\ \hline
08.03.2016 & 1.0 & Osnovna verzija & Nemanja Stojoski \\ \hline
 &  &  &  \\ \hline
 &  &  &  \\ \hline
 &  &  &  \\ \hline
 &  &  &  \\ \hline
 &  &  &  \\ \hline
\end{tabular}
\end{center}

\end{document}