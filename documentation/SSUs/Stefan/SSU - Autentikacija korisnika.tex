\documentclass[11pt,a4paper]{article}
\usepackage[utf8]{inputenc}
\usepackage[T1]{fontenc}
\usepackage[serbian]{babel}
\usepackage[table,xcdraw]{xcolor}
\usepackage{array}
\usepackage{graphicx}
\usepackage{hyperref}
\hypersetup{
	colorlinks,
	citecolor=black,
	filecolor=black,
	linkcolor=black,
	urlcolor=black
}

\renewcommand*\contentsname{Sadržaj}

\begin{document}

\begin{titlepage}

\centering
\textnormal{\large Elektrotehnički fakultet Univerziteta u Beogradu}\\[0.1cm]
\textnormal{\large Prinicipi softverskog inženjerstva}\\[3cm]

\textnormal{\normalsize - Platforma za organizovanje volonterskih akcija -}\\\vspace{-5mm}
\rule{\textwidth}{0.4pt}
{\huge \bfseries Specifikacija scenarija upotrebe:\\ 
AUTENTIKACIJA KORISNIKA\par}\vspace{-1mm}
\rule{\textwidth}{0.4pt}\\\vspace{1mm}
\textnormal{\large Autor: Stefan Bajić}\\[6cm]

\includegraphics[scale=0.5]{logo.jpg}\\
\vfill
\textnormal{\normalsize Verzija 1.0}\\

\end{titlepage}

\tableofcontents

\newpage

\section{Uvod}
\subsection{Rezime}
Definisanje scenarija upotrebe pri prijavi korisnika na sistem, sa primerima odgovarajućih HTML stranica.
\subsection{Namena dokumenta i ciljne grupe}
Dokument će koristiti svi članovi projektnog tima u razvoju projekta i testiranju, a može se koristiti i pri pisanju uputstva za upotrebu.
\subsection{Reference}
\begin{enumerate}
    \item Projektni zadatak
    \item Uputstvo za pisanje specifikacije scenarija upotrebe funkcionalnosti
\end{enumerate}
\subsection{Otvorena pitanja}
\begin{center}
\begin{tabular}{| >{\centering\arraybackslash}m{1.9cm} | >{\centering\arraybackslash}m{4.9cm} | >{\centering\arraybackslash}m{4.9cm} |}
\hline
\rowcolor[HTML]{000000} 
{\color[HTML]{FFFFFF} Redni broj } & {\color[HTML]{FFFFFF} Opis } & {\color[HTML]{FFFFFF} Rešenje } \\ \hline
 &  &  \\ \hline
 &  &  \\ \hline
 &  &  \\ \hline
 &  &  \\ \hline
\end{tabular}
\end{center}

\newpage

\section{Scenario registracije korisnika}
\subsection{Kratak opis}
Gost sistema može, ukoliko želi i ukoliko ima nalog, da se prijavi na sistem unošenjem ličnih
podataka (korisničko ime i lozinka). Vrši se provera čitanjem iz baze podataka i ukoliko se podaci poklapaju korisnik se uspešno prijavljuje i dobija određene dodatne slobode.
\subsection{Tok događaja}
\subsubsection{\underline{Korisnik se uspešno prijavljuje na sistem}}
\begin{enumerate}
    \item Korisnik odabirom polja \textit{Login} dolazi na stranicu za prijavu
    \item Korisnik unosi korisničko ime i lozinku
    \item Korisnik pritiska dugme \textit{Login}
    \item Sistem proverava uneto korisničko ime i lozinku sa već postojećim u bazi podataka i ukoliko se poklapaju
    \item Sistem prebacuje korisnika na početnu stranicu
\end{enumerate}

\subsubsection{Nepostojeća kombinacija korisničkog imena i lozinke}
\begin{enumerate}
    \item Akcije 1-3 iz scenarija 2.2.1 su iste
    \item Sistem proverava uneto korisničko ime i lozinku sa već postojećim u bazi podataka i prikazuje poruku o pogrešno unetoj lozinci ili korisničkom imenu
\end{enumerate}

\subsubsection{Korisnik nije pravilno ili uopšte popunio jedno ili oba polja potrebna za prijavu}
\begin{enumerate}
    \item Akcija 1 iz scenarija 2.2.1 je ista
    \item Korisnik ne popunjava pravilno ili uopšte jedno ili oba polja potrebna za prijavu
    \item Korisnik pritiska dugme \textit{Login}
    \item Sistem obaveštava korisnika o njegovoj grešci odgovarajućom porukom
\end{enumerate}
\subsection{Posebni zahtevi}
Nema
\subsection{Preduslovi}
Korisnik nije ulogovan na sistem.
\subsection{Posledice}
U slučaju uspeha korisnik je prijavljen na sistem.
\newpage

\section{Spisak izmena}
\begin{center}
\begin{tabular}{| >{\centering\arraybackslash}m{2cm} | >{\centering\arraybackslash}m{1.3cm} | >{\centering\arraybackslash}m{4.2cm} | >{\centering\arraybackslash}m{4.2cm} |}
\hline
\rowcolor[HTML]{000000} 
{\color[HTML]{FFFFFF} Datum } & {\color[HTML]{FFFFFF} Verzija } & {\color[HTML]{FFFFFF} Opis izmene } & {\color[HTML]{FFFFFF} Autor } \\ \hline
14.03.2016 & 1.0 & Osnovna verzija & Stefan Bajić \\ \hline
 &  &  &  \\ \hline
 &  &  &  \\ \hline
 &  &  &  \\ \hline
 &  &  &  \\ \hline
 &  &  &  \\ \hline
\end{tabular}
\end{center}

\end{document}
