\documentclass[11pt,a4paper]{article}
\usepackage[utf8]{inputenc}
\usepackage[T1]{fontenc}
\usepackage[serbian]{babel}
\usepackage[table,xcdraw]{xcolor}
\usepackage{array}
\usepackage{graphicx}
\usepackage{hyperref}
\hypersetup{
	colorlinks,
	citecolor=black,
	filecolor=black,
	linkcolor=black,
	urlcolor=black
}

\renewcommand*\contentsname{Sadržaj}

\begin{document}

\begin{titlepage}

\centering
\textnormal{\large Elektrotehnički fakultet Univerziteta u Beogradu}\\[0.1cm]
\textnormal{\large Prinicipi softverskog inženjerstva}\\[3cm]

\textnormal{\normalsize - Platforma za organizovanje volonterskih akcija -}\\\vspace{-5mm}
\rule{\textwidth}{0.4pt}
{\huge \bfseries Specifikacija scenarija upotrebe:\\ 
IZMENA DOGAĐAJA\par}\vspace{-1mm}
\rule{\textwidth}{0.4pt}\\\vspace{1mm}
\textnormal{\large Autor: Ana Peško}\\[6cm]

\includegraphics[scale=0.5]{logo.jpg}\\
\vfill
\textnormal{\normalsize Verzija 1.0}\\

\end{titlepage}

\tableofcontents

\newpage

\section{Uvod}
\subsection{Rezime}
Definisanje scenarija upotrebe pri izmeni događaja, sa primerima odgovarajućih HTML stranica.
\subsection{Namena dokumenta i ciljne grupe}
Dokument će koristiti svi članovi projektnog tima u razvoju projekta i testiranju, a može se koristiti i pri pisanju uputstva za upotrebu.
\subsection{Reference}
\begin{enumerate}
    \item Projektni zadatak
    \item Uputstvo za pisanje specifikacije scenarija upotrebe funkcionalnosti
\end{enumerate}
\subsection{Otvorena pitanja}
\begin{center}
\begin{tabular}{| >{\centering\arraybackslash}m{1.9cm} | >{\centering\arraybackslash}m{4.9cm} | >{\centering\arraybackslash}m{4.9cm} |}
\hline
\rowcolor[HTML]{000000} 
{\color[HTML]{FFFFFF} Redni broj } & {\color[HTML]{FFFFFF} Opis } & {\color[HTML]{FFFFFF} Rešenje } \\ \hline
 1 & Šta se dešava ukoliko korisnik ne izmeni ništa u događaju a klikne na dugme \textit{Save}? & \\ \hline
 &  &  \\ \hline
 &  &  \\ \hline
 &  &  \\ \hline
\end{tabular}
\end{center}

\newpage

\section{Scenario izmene događaja}
\subsection{Kratak opis}
Tvorac događaja u svakom trenutku može vršiti proizvoljne modifikacije teksta stranice događaja kao i dodavanje novih slika a potom se te izmene šalju određenom moderatoru radi provere.
\subsection{Tok događaja}
\subsubsection{\underline{Korisnik uspešno menja događaj}}
\begin{enumerate}
    \item Korisnik odabirom polja \textit{Home} dolazi na stranicu za pregled ličnih i pohađanih događaja
    \item Korisnik pregleda događaje koje trenutno organizuje (lične događaje)
    \item Korisnik odabirom događaja koji želi da promeni (klikom na ikonicu za izmenu) dolazi na stranicu za kreiranje događaja popunjenu trenutnim informacijama o događaju
    \item Korisnik menja željene informacije o događaju
    \item Korisnik klikom na dugme \textit{Save Changes} čuva izmene
    \item Sistem prosleđuje zahtev za izmenom moderatoru na odobravanje i vraća korisnika na stranicu za pregled ličnih i pohađanih događaja
\end{enumerate}

\subsubsection{Korisnik bira da ne promeni događaj}
\begin{enumerate}
    \item Akcije 1-4 iz scenarija 2.2.1 su iste
    \item Korisnik klikom na dugme \textit{Discard Changes} odustaje od izmene događaja
    \item Sistem vraća korisnika na stranicu za pregled ličnih i pohađanih događaja bez prosleđivanja zahteva za izmenom moderatoru na odobravanje
\end{enumerate}

\subsection{Posebni zahtevi}
Nema.
\subsection{Preduslovi}
Korisnik je ulogovan na sistem i organizator je bar jednog aktuelnog događaja.
\subsection{Posledice}
U slučaju uspešnog scenarija, izmene događaja su poslate moderatoru, a u suprotnom izmene se ne šalju.

\newpage

\section{Spisak izmena}
\begin{center}
\begin{tabular}{| >{\centering\arraybackslash}m{2cm} | >{\centering\arraybackslash}m{1.3cm} | >{\centering\arraybackslash}m{4.2cm} | >{\centering\arraybackslash}m{4.2cm} |}
\hline
\rowcolor[HTML]{000000} 
{\color[HTML]{FFFFFF} Datum } & {\color[HTML]{FFFFFF} Verzija } & {\color[HTML]{FFFFFF} Opis } & {\color[HTML]{FFFFFF} Autor } \\ \hline
12.03.2016 & 1.0 & Osnovna verzija & Ana Peško \\ \hline
 &  &  &  \\ \hline
 &  &  &  \\ \hline
 &  &  &  \\ \hline

\end{tabular}
\end{center}

\end{document}