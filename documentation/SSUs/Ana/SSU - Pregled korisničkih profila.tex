\documentclass[11pt,a4paper]{article}
\usepackage[utf8]{inputenc}
\usepackage[T1]{fontenc}
\usepackage[serbian]{babel}
\usepackage[table,xcdraw]{xcolor}
\usepackage{array}
\usepackage{graphicx}
\usepackage{hyperref}
\hypersetup{
	colorlinks,
	citecolor=black,
	filecolor=black,
	linkcolor=black,
	urlcolor=black
}

\renewcommand*\contentsname{Sadržaj}

\begin{document}

\begin{titlepage}

\centering
\textnormal{\large Elektrotehnički fakultet Univerziteta u Beogradu}\\[0.1cm]
\textnormal{\large Prinicipi softverskog inženjerstva}\\[3cm]

\textnormal{\normalsize - Platforma za organizovanje volonterskih akcija -}\\\vspace{-5mm}
\rule{\textwidth}{0.4pt}
{\huge \bfseries Specifikacija scenarija upotrebe:\\ 
PREGLED KORISNIČKIH PROFILA\par}\vspace{-1mm}
\rule{\textwidth}{0.4pt}\\\vspace{1mm}
\textnormal{\large Autor: Ana Peško}\\[6cm]

\includegraphics[scale=0.5]{logo.jpg}\\
\vfill
\textnormal{\normalsize Verzija 1.0}\\

\end{titlepage}

\tableofcontents

\newpage

\section{Uvod}
\subsection{Rezime}
Definisanje scenarija upotrebe pri pregledu profila registrovanih korisnika, sa primerima odgovarajućih HTML stranica.
\subsection{Namena dokumenta i ciljne grupe}
Dokument će koristiti svi članovi projektnog tima u razvoju projekta i testiranju, a može se koristiti i pri pisanju uputstva za upotrebu.
\subsection{Reference}
\begin{enumerate}
    \item Projektni zadatak
    \item Uputstvo za pisanje specifikacije scenarija upotrebe funkcionalnosti
\end{enumerate}
\subsection{Otvorena pitanja}
\begin{center}
\begin{tabular}{| >{\centering\arraybackslash}m{1.9cm} | >{\centering\arraybackslash}m{4.9cm} | >{\centering\arraybackslash}m{4.9cm} |}
\hline
\rowcolor[HTML]{000000} 
{\color[HTML]{FFFFFF} Redni broj } & {\color[HTML]{FFFFFF} Opis } & {\color[HTML]{FFFFFF} Rešenje } \\ \hline
1. & Da li korisnik čija je profilna strana može da menja informacije na njoj? & \\ \hline
 &  &  \\ \hline
 &  &  \\ \hline
 &  &  \\ \hline
\end{tabular}
\end{center}

\newpage

\section{Scenario pregleda korisničkog profila}
\subsection{Kratak opis}
Osnovne informacije korisničkih profila su javne a korisnici po želji mogu omogućiti prikaz dodatnih.

\subsection{Tok događaja}
\subsubsection{\underline{Korisnik pregleda korisnički profil}}
\begin{enumerate}
    \item Korisnik dolazi na stranu za pregled nečijeg korisničkog profila
    \item Korisnik pregleda korisnički profil i informacije na njemu
\end{enumerate}

\subsection{Posebni zahtevi}
Nema.
\subsection{Preduslovi}
Nema.
\subsection{Posledice}
Nema.

\newpage

\section{Spisak izmena}
\begin{center}
\begin{tabular}{| >{\centering\arraybackslash}m{2cm} | >{\centering\arraybackslash}m{1.3cm} | >{\centering\arraybackslash}m{4.2cm} | >{\centering\arraybackslash}m{4.2cm} |}
\hline
\rowcolor[HTML]{000000} 
{\color[HTML]{FFFFFF} Datum } & {\color[HTML]{FFFFFF} Verzija } & {\color[HTML]{FFFFFF} Opis } & {\color[HTML]{FFFFFF} Autor } \\ \hline
10.03.2016 & 1.0 & Osnovna verzija & Ana Peško \\ \hline
 &  &  &  \\ \hline
 &  &  &  \\ \hline
 &  &  &  \\ \hline

\end{tabular}
\end{center}

\end{document}