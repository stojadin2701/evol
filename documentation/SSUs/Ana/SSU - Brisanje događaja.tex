\documentclass[11pt,a4paper]{article}
\usepackage[utf8]{inputenc}
\usepackage[T1]{fontenc}
\usepackage[serbian]{babel}
\usepackage[table,xcdraw]{xcolor}
\usepackage{array}
\usepackage{graphicx}
\usepackage{hyperref}
\hypersetup{
	colorlinks,
	citecolor=black,
	filecolor=black,
	linkcolor=black,
	urlcolor=black
}

\renewcommand*\contentsname{Sadr�aj}

\begin{document}

\begin{titlepage}

\centering
\textnormal{\large Elektrotehnicki fakultet Univerziteta u Beogradu}\\[0.1cm]
\textnormal{\large Prinicipi softverskog in�enjerstva}\\[3cm]

\textnormal{\normalsize - Platforma za organizovanje volonterskih akcija -}\\\vspace{-5mm}
\rule{\textwidth}{0.4pt}
{\huge \bfseries Specifikacija scenarija upotrebe:\\ 
BRISANJE DOGAdAJA\par}\vspace{-1mm}
\rule{\textwidth}{0.4pt}\\\vspace{1mm}
\textnormal{\large Autor: Ana Pe�ko}\\[6cm]

\includegraphics[scale=0.5]{logo.jpg}\\
\vfill
\textnormal{\normalsize Verzija 1.0}\\

\end{titlepage}

\tableofcontents

\newpage

\section{Uvod}
\subsection{Rezime}
Definisanje scenarija upotrebe pri brisanju dogadaja, sa primerima odgovarajucih HTML stranica.
\subsection{Namena dokumenta i ciljne grupe}
Dokument ce koristiti svi clanovi projektnog tima u razvoju projekta i testiranju, a mo�e se koristiti i pri pisanju uputstva za upotrebu.
\subsection{Reference}
\begin{enumerate}
    \item Projektni zadatak
    \item Uputstvo za pisanje specifikacije scenarija upotrebe funkcionalnosti
\end{enumerate}
\subsection{Otvorena pitanja}
\begin{center}
\begin{tabular}{| >{\centering\arraybackslash}m{1.9cm} | >{\centering\arraybackslash}m{4.9cm} | >{\centering\arraybackslash}m{4.9cm} |}
\hline
\rowcolor[HTML]{000000} 
{\color[HTML]{FFFFFF} Redni broj } & {\color[HTML]{FFFFFF} Opis } & {\color[HTML]{FFFFFF} Re�enje } \\ \hline
 &  & \\ \hline
 &  &  \\ \hline
 &  &  \\ \hline
 &  &  \\ \hline
\end{tabular}
\end{center}

\newpage

\section{Scenario brisanja dogadaja}
\subsection{Kratak opis}
Registrovani korisnik ima mogucnost brisanja dogadaja u svakom trenutku.
\subsection{Tok dogadaja}
\subsubsection{\underline{Korisnik uspe�no bri�e dogadaj}}
\begin{enumerate}
    \item Korisnik odabirom polja \textit{Home} dolazi na stranicu za pregled licnih i pohadanih dogadaja
    \item Korisnik pregleda dogadaje koje trenutno organizuje (licne dogadaje)
    \item Korisniku se odabirom dogadaja koji �eli da izbri�e (klikom na ikonicu za brisanje) prikazuje pitanje da li sigurno �eli da obri�e taj dogadaj
    \item Korisnik bira opciju \textit{OK}
    \item Sistem uklanja dogadaj iz baze podataka i vraca korisnika na stranicu za pregled licnih i pohadanih dogadaja
\end{enumerate}

\subsubsection{Korisnik bira da ne obri�e dogadaj}
\begin{enumerate}
    \item Akcije 1-3 iz scenarija 2.2.1 su iste
    \item Korisnik bira opciju \textit{Cancel}
    \item Sistem vraca korisnika na stranicu za pregled licnih i pohadanih dogadaja bez uklanjanja dogadaja iz baze podataka
\end{enumerate}

\subsection{Posebni zahtevi}
Nema.
\subsection{Preduslovi}
Korisnik je ulogovan na sistem i organizator je bar jednog aktuelnog dogadaja.
\subsection{Posledice}
U slucaju uspe�nog toka scenarija, dogadaj je uklonjen iz sistema, u suprotnom dogadaj se i dalje nalazi u evidenciji sistema.

\newpage

\section{Spisak izmena}
\begin{center}
\begin{tabular}{| >{\centering\arraybackslash}m{2cm} | >{\centering\arraybackslash}m{1.3cm} | >{\centering\arraybackslash}m{4.2cm} | >{\centering\arraybackslash}m{4.2cm} |}
\hline
\rowcolor[HTML]{000000} 
{\color[HTML]{FFFFFF} Datum } & {\color[HTML]{FFFFFF} Verzija } & {\color[HTML]{FFFFFF} Opis } & {\color[HTML]{FFFFFF} Autor } \\ \hline
12.03.2016 & 1.0 & Osnovna verzija & Ana Pe�ko \\ \hline
 &  &  &  \\ \hline
 &  &  &  \\ \hline
 &  &  &  \\ \hline

\end{tabular}
\end{center}

\end{document}