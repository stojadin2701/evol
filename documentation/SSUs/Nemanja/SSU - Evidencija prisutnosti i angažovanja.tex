\documentclass[11pt,a4paper]{article}
\usepackage[utf8]{inputenc}
\usepackage[T1]{fontenc}
\usepackage[serbian]{babel}
\usepackage[table,xcdraw]{xcolor}
\usepackage{array}
\usepackage{graphicx}
\usepackage{hyperref}
\hypersetup{
	colorlinks,
	citecolor=black,
	filecolor=black,
	linkcolor=black,
	urlcolor=black
}

\renewcommand*\contentsname{Sadržaj}

\begin{document}

\begin{titlepage}

\centering
\textnormal{\large Elektrotehnički fakultet Univerziteta u Beogradu}\\[0.1cm]
\textnormal{\large Prinicipi softverskog inženjerstva}\\[3cm]

\textnormal{\normalsize - Platforma za organizovanje volonterskih akcija -}\\\vspace{-5mm}
\rule{\textwidth}{0.4pt}
{\huge \bfseries Specifikacija scenarija upotrebe:\\ 
EVIDENCIJA PRISUTNOSTI I ANGAŽOVANJA\par}\vspace{-1mm}
\rule{\textwidth}{0.4pt}\\\vspace{1mm}
\textnormal{\large Autor: Nemanja Stojoski}\\[6cm]

\includegraphics[scale=0.5]{logo.jpg}\\
\vfill
\textnormal{\normalsize Verzija 1.0}\\

\end{titlepage}

\tableofcontents

\newpage

\section{Uvod}
\subsection{Rezime}
Definisanje scenarija upotrebe pri evidenciji prisutnosti i angažovanja korisnika na nekom događaju, sa primerima odgovarajućih HTML stranica.
\subsection{Namena dokumenta i ciljne grupe}
Dokument će koristiti svi članovi projektnog tima u razvoju projekta i testiranju, a može se koristiti i pri pisanju uputstva za upotrebu.
\subsection{Reference}
\begin{enumerate}
    \item Projektni zadatak
    \item Uputstvo za pisanje specifikacije scenarija upotrebe funkcionalnosti
\end{enumerate}
\subsection{Otvorena pitanja}
\begin{center}
\begin{tabular}{| >{\centering\arraybackslash}m{1.9cm} | >{\centering\arraybackslash}m{4.9cm} | >{\centering\arraybackslash}m{4.9cm} |}
\hline
\rowcolor[HTML]{000000} 
{\color[HTML]{FFFFFF} Redni broj } & {\color[HTML]{FFFFFF} Opis } & {\color[HTML]{FFFFFF} Rešenje } \\ \hline
1 & Da li učesnici događaja treba da budu obavešteni o njihovom angažovanju? & \\ \hline
 &  &  \\ \hline
 &  &  \\ \hline
 &  &  \\ \hline
\end{tabular}
\end{center}

\newpage

\section{Scenario evidencije prisutnosti i angažovanja}
\subsection{Kratak opis}
Registrovani korisnik koji je organizovao događaj ima obavezu da vodi evidenciju tokom održavanja i posle završetka događaja. Nakon završetka događaja organizator je u obavezi da popuni odgovarajuću elektronsku formu radi raspodele virtuelnih bedževa zaslužnim učesnicima.
\subsection{Tok događaja}
\subsubsection{\underline{Organizator događaja uspešno evidentira učesnike}}
\begin{enumerate}
    \item Organizator događaja odabirom polja \textit{Home} dolazi na stranicu za pregled svog naloga
    \item Organizatoru se odabirom opcije \textit{MyEvents} prikazuju svi događaji koje trenutno organizuje
    \item Organizator odabirom tekućeg ili završenog događaja za koji hoće da evidentira prisutnost (klikom na ikonicu za evidentiranje) dolazi na stranicu za evidentiranje učesnika
    \item Organizator prolazi kroz listu prijavljenih učesnika za dati događaj i evidentira njihovu prisutnost, ocenjuje angažovanje i opciono daje tekstualni komentar
\end{enumerate}

\subsection{Posebni zahtevi}
Nema
\subsection{Preduslovi}
Korisnik je ulogovan na sistem i organizator je događaja koji je u toku ili se završio a učesnici nisu evidentirani
\subsection{Posledice}
Učesnici događaja su evidentirani i njihovo angažovanje je ocenjeno.
\newpage

\section{Spisak izmena}
\begin{center}
\begin{tabular}{| >{\centering\arraybackslash}m{2cm} | >{\centering\arraybackslash}m{1.3cm} | >{\centering\arraybackslash}m{4.2cm} | >{\centering\arraybackslash}m{4.2cm} |}
\hline
\rowcolor[HTML]{000000} 
{\color[HTML]{FFFFFF} Datum } & {\color[HTML]{FFFFFF} Verzija } & {\color[HTML]{FFFFFF} Opis } & {\color[HTML]{FFFFFF} Autor } \\ \hline
14.03.2016 & 1.0 & Osnovna verzija & Nemanja Stojoski \\ \hline
 &  &  &  \\ \hline
 &  &  &  \\ \hline
 &  &  &  \\ \hline

\end{tabular}
\end{center}

\end{document}
