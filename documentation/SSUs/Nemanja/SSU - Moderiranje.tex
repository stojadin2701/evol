\documentclass[11pt,a4paper]{article}
\usepackage[utf8]{inputenc}
\usepackage[T1]{fontenc}
\usepackage[serbian]{babel}
\usepackage[table,xcdraw]{xcolor}
\usepackage{array}
\usepackage{graphicx}
\usepackage{hyperref}
\hypersetup{
	colorlinks,
	citecolor=black,
	filecolor=black,
	linkcolor=black,
	urlcolor=black
}

\renewcommand*\contentsname{Sadržaj}

\begin{document}

\begin{titlepage}

\centering
\textnormal{\large Elektrotehnički fakultet Univerziteta u Beogradu}\\[0.1cm]
\textnormal{\large Prinicipi softverskog inženjerstva}\\[3cm]

\textnormal{\normalsize - Platforma za organizovanje volonterskih akcija -}\\\vspace{-5mm}
\rule{\textwidth}{0.4pt}
{\huge \bfseries Specifikacija scenarija upotrebe:\\ 
MODERIRANJE\par}\vspace{-1mm}
\rule{\textwidth}{0.4pt}\\\vspace{1mm}
\textnormal{\large Autor: Nemanja Stojoski}\\[6cm]

\includegraphics[scale=0.5]{logo.jpg}\\
\vfill
\textnormal{\normalsize Verzija 1.0}\\

\end{titlepage}

\tableofcontents

\newpage

\section{Uvod}
\subsection{Rezime}
Definisanje scenarija upotrebe pri moderiranju događaja, sa primerima odgovarajućih HTML stranica.
\subsection{Namena dokumenta i ciljne grupe}
Dokument će koristiti svi članovi projektnog tima u razvoju projekta i testiranju, a može se koristiti i pri pisanju uputstva za upotrebu.
\subsection{Reference}
\begin{enumerate}
    \item Projektni zadatak
    \item Uputstvo za pisanje specifikacije scenarija upotrebe funkcionalnosti
\end{enumerate}
\subsection{Otvorena pitanja}
\begin{center}
\begin{tabular}{| >{\centering\arraybackslash}m{1.9cm} | >{\centering\arraybackslash}m{4.9cm} | >{\centering\arraybackslash}m{4.9cm} |}
\hline
\rowcolor[HTML]{000000} 
{\color[HTML]{FFFFFF} Redni broj } & {\color[HTML]{FFFFFF} Opis } & {\color[HTML]{FFFFFF} Rešenje } \\ \hline
1 & Da li moderator može da šalje povratne poruke korisnicima? & \\ \hline
 &  &  \\ \hline
 &  &  \\ \hline
 &  &  \\ \hline
\end{tabular}
\end{center}

\newpage

\section{Scenario moderiranja}
\subsection{Kratak opis}
Moderatori sajta preko njima specifičnog interfejsa mogu odobravati događaje, izmene događaja kao i imati pregled svih korisničkih poruka dobijenih preko stranice za feedback.
\subsection{Tok događaja}
\subsubsection{\underline{Moderator odobrava događaj}}
\begin{enumerate}
    \item Moderator odabirom polja \textit{Mod} dolazi na stranicu za moderiranje
    \item Moderatoru se u delu stranice \textit{Events to review} prikazuje lista događaja koje treba da odobri ili ne odobri 
    \item Moderator nakon pregledanja informacija o događaju odlučuje da želi da ga odobri i klikne na dugme za odobravanje događaja
    \item Sistem uklanja događaj iz moderatorove liste za odobravanje i događaj postaje javno vidljiv.
\end{enumerate}

\subsubsection{\underline{Moderator ne odobrava događaj}}
\begin{enumerate}
    \item Akcije 1-2 iz scenarija 2.2.1 su iste
    \item Moderator nakon pregledanja informacija o događaju odlučuje da ne želi da ga odobri i klikne na dugme za uklanjanje događaja
    \item Sistem uklanja događaj iz moderatorove liste kao i čuva ga kao neodobreni događaj.
\end{enumerate}

\subsubsection{\underline{Moderator čita poruke korisnika}}
\begin{enumerate}
    \item Akcija 1 iz scenarija 2.2.1 je ista
    \item Moderatoru se u delu stranice \textit{User feedback} prikazuje spisak poruka koje su poslali korisnici kao feedback
    \item Moderator čita poruke korisnika
    \item Moderator klikom na dugme može ukloniti poruku
\end{enumerate}

\newpage

\subsection{Posebni zahtevi}
Nema
\subsection{Preduslovi}
Moderator je ulogovan na sistem.
\subsection{Posledice}
U zavisnosti od scenarija događaji su odobreni ili neodobreni, ili su poruke korisnika pročitane
\newpage

\section{Spisak izmena}
\begin{center}
\begin{tabular}{| >{\centering\arraybackslash}m{2cm} | >{\centering\arraybackslash}m{1.3cm} | >{\centering\arraybackslash}m{4.2cm} | >{\centering\arraybackslash}m{4.2cm} |}
\hline
\rowcolor[HTML]{000000} 
{\color[HTML]{FFFFFF} Datum } & {\color[HTML]{FFFFFF} Verzija } & {\color[HTML]{FFFFFF} Opis } & {\color[HTML]{FFFFFF} Autor } \\ \hline
12.03.2016 & 1.0 & Osnovna verzija & Nemanja Stojoski \\ \hline
 &  &  &  \\ \hline
 &  &  &  \\ \hline
 &  &  &  \\ \hline

\end{tabular}
\end{center}

\end{document}